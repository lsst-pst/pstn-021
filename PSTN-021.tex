\documentclass[modern]{aastex62}

% lsstdoc documentation: https://lsst-texmf.lsst.io/lsstdoc.html
% GENERATED FILE -- edit this in the Makefile
\newcommand{\lsstDocType}{PSTN}
\newcommand{\lsstDocNum}{021}
\newcommand{\vcsRevision}{e6ee10c-dirty}
\newcommand{\vcsDate}{2024-04-12}


% Package imports go here.

\usepackage{tcolorbox}

% Local commands go here.

\newcommand{\note}[1]{
   \begin{tcolorbox}[colback=red!5!white, colframe=red!75!black]
      #1
   \end{tcolorbox}
}



\newcommand{\docRef}{PSTN-021}
\newcommand{\docUpstreamLocation}{\url{https://github.com/lsst-pst/pstn-021}}


\begin{document}
%% DO NOT EDIT THIS FILE. IT IS GENERATED FROM db2authors.py"
%% Regenerate using:
%%    python $LSST_TEXMF_DIR/bin/db2authors.py > authors.tex


\author[0000-0001-8018-5348]{Eric~C.~Bellm}
\affiliation{University of Washington, Dept.\ of Astronomy, Box 351580, Seattle, WA 98195, USA}

\author[0000-0001-8708-251X]{Ian~S.~Sullivan}
\affiliation{University of Washington, Dept.\ of Astronomy, Box 351580, Seattle, WA 98195, USA}


\date{\today}
\title{LSST Prompt Processing}
\hypersetup{pdftitle={\@title}, pdfauthor={\@author}, pdfkeywords={\@keywords}}


\begin{abstract}

Describes the LSST Alert Production pipelines, excecution environment, data products, and early performance. 

\end{abstract}



\section{Introduction}

\note{See discussion on common material and timing considerations in \cite{PSTN-020}.}

\note{This outline proceeds assuming this paper is being produced at the end of commissioning/early operations, and some form of routine alert production is happening in operations based on commissioning-derived or incremental templates.

If only a small amount of LSSTCam commissioning data is available and is being released, only a minimal paper decribing the pipelines used to process that data would useful, and a more complete AP reference should come later.}

\note{Information about the processing environment is presumed to be covered in \cite{PSTN-018}.
}

\note{SS Processing falls under Prompt Processing but detailed discussion is deferred to \cite{PSTN-025}.}


\section{Pipeline}
\note{following LDM-151}
\subsection{Single Frame Processing}

\note{Construction of calibration products assumed described in \cite{PSTN-026}.}

ISR, PSF and Background fitting, photometric and astrometric calibration.  Likely major commonalities with DRP.

\subsection{Template Generation}

\note{Pre-DR1 this may require description independent of DRP.}

High-level discussion of DCR correction could go here.

\subsection{Image Differencing}

\subsection{Source Detection}

Including discussion of point source, dipole, and streak detection on difference images.  

\note{Algorithmic basics could be discussed elsewhere}

\subsection{Spuriousness scoring}

Algorithm, training, and performance of the ML spuriousness score

\subsection{AP Source Association}

Algorithms and interface considerations with the PPDB

\note{does \cite{PSTN-045} describe the APDB/PPDB implementation?}

\subsection{Alert Generation}

Formats \& contents

\subsection{Alert Distribution}

Mechanisms, connections to community brokers

\subsection{Alert Filtering Service}

\subsection{Forced Photometry}


\section{Prompt Data Products}


Summary of relevant aspects of the DPDD \citep{LSE-163}, including latency considerations and user access.

\begin{itemize}
  \item Images
  \item Prompt Catalogs and the PPDB
  \item Alerts
\end{itemize}




\appendix
% Remove this when you strart your paper

{\bf Initial paper list added here for reference.}

``Editor'' is a responsible team leader but not necessarily the person who will do most of
the required work, or who will eventually become the first author. Both issues will be
handled by individual teams.

\begin{verbatim}

domain: Telescope & Site
editor: Jeff Barr
title: Overview of the LSST Telescope

domain: Telescope & Site
editor: Sandrine Thomas
title: Performance of the LSST Telescope

domain: Telescope & Site
editor: Lynne Jones
title: The LSST Scheduler Overview and Performance

domain: Telescope & Site
editor: Bo Xin
title: Performance of the LSST Active Optics System

domain: Telescope & Site
editor: Tiago Ribeiro
title: LSST Observing System Software Architecture

domain: Camera
editor: Justin Wolfe
title: LSST Camera Optics

domain: Camera
editor: Chris Stubbs
title: LSST Camera Rafts

domain: Camera
editor: Steve Ritz
title: LSST Camera Cryostat

domain: Camera
editor: Ralph Schindler
title: LSST Camera Refrigeration

domain: Camera
editor: Steve Ritz
title: LSST Camera Body and Mechanisms

domain: Camera
editor: Mark Huffer and Tony Johnson
title: LSST Camera Control System and DAQ

domain: Camera
editor: Tim Bond and Aaron Rodman
title: LSST Camera Integration and Tests

domain: Data Management
editor: Leanne Guy
title: Overview of LSST Data Management

domain: Data Management
editor: Michelle Butler
title: LSST Data Facility

domain: Data Management
editor: Tim Jenness
title: LSST Data Management Software System

domain: Data Management
editor: Jim Bosch
title: LSST Data Release Processing

domain: Data Management
editor: Eric Bellm
title: LSST Prompt Data Products

domain: Data Management
editor: Gregory Dubois-Felsmann
title: LSST Science Platform

domain: Data Management
editor: Simon Krughoff
title: LSST Data Management Quality Assurance and Reliability Engineering

domain: Data Management
editor: Leanne Guy (with likely delegation to new DM V&V Scientist)
title: LSST Data Management System Verification and Validation

domain: Data Management
editor: Mario Juric
title: LSST Moving Object Processing

domain: Data Management
editor: Robert Lupton
title: LSST Calibration Strategy and Pipelines

domain: Calibration
editor: Patrick Ingraham
title:  Performance of the LSST Calibration Systems

domain: Calibration
editor: Patrick Ingraham
title: Atmospheric Properties with the LSST Auxiliary Telescope

domain: EPO
editor: Amanda Bauer
title: Overview of LSST Education and Public Outreach

domain: EPO
editor: Ardis Herrold
title: LSST Formal Education Program

domain: EPO
editor: Amanda Bauer
title: LSST EPO: The User Feedback

domain: Commissioning
editor: Chuck Claver
title: LSST Observatory System Operations Readiness Report

domain: Commissioning
editor: Bo Xin
title: Performance of Delivered LSST System

domain: Commissioning
editor: Chuck Claver
title: Active Optics Performance with LSST Commissiong Camera

domain: Commissioning
editor: Chuck Claver
title: LSST Active Optics Performance with the LSST Science Camera

domain: Commissioning
editor: Brian Stalder
title: Integration, Test and Commissioning Results from LSST Commissiong Camera

domain: Commissioning
editor: Kevin Reil
title: LSST Camera Instrumental Signature Characterization, Calibration and Removal

domain: Commissioning
editor: Patrick Hascal
title: Installation and Performance of the LSST Camera Refrigeration System

domain: Commissioning
editor: Andy Connolly
title: Science Validation of LSST Alert Processing

domain: Commissioning
editor: Keith Bechtol
title: Science Validation of LSST Data Release Processing

domain: Commissioning
editor: Michael Reuter
title: Tracking of LSST System Performance with Continuous Integration Methods

domain: Commissioning
editor: Chuck Claver
title: The LSST Science Platform as a Commissioning Tool

domain: Commissioning
editor: Chuck Claver
title: Commissioning Science Data Quality Analysis Tools, Methods and Procedures

domain: Commissioning
editor: Lynne Jones
title: Performance Verification of the LSST Survey Scheduler


\end{verbatim}

% Include all the relevant bib files.
% https://lsst-texmf.lsst.io/lsstdoc.html#bibliographies
\section{References} \label{sec:bib}
\bibliographystyle{yahapj}
\bibliography{local,lsst,lsst-dm,refs_ads,refs,books}

% Make sure lsst-texmf/bin/generateAcronyms.py is in your path
\section{Acronyms} \label{sec:acronyms}
\addtocounter{table}{-1}
\begin{longtable}{p{0.145\textwidth}p{0.8\textwidth}}\hline
\textbf{Acronym} & \textbf{Description}  \\\hline

AP & Alert Production \\\hline
APDB & Alert Production DataBase \\\hline
DCR & Differential Chromatic Refraction \\\hline
DPDD & Data Product Definition Document \\\hline
DRP & Data Release Production \\\hline
ISR & Instrument Signal Removal \\\hline
LSE & LSST Systems Engineering (Document Handle) \\\hline
LSST & Large Synoptic Survey Telescope \\\hline
PPDB & Prompt Products DataBase \\\hline
PSF & Point Spread Function \\\hline
PSTN & Project Science Technical Note \\\hline
\end{longtable}


\end{document}
