\documentclass[modern, twocolumn]{aastex631}

% lsstdoc documentation: https://lsst-texmf.lsst.io/lsstdoc.html
% GENERATED FILE -- edit this in the Makefile
\newcommand{\lsstDocType}{PSTN}
\newcommand{\lsstDocNum}{021}
\newcommand{\vcsRevision}{e6ee10c-dirty}
\newcommand{\vcsDate}{2024-04-12}


% Package imports go here.

\usepackage{tcolorbox}
\usepackage[colorinlistoftodos]{todonotes}
\let\Oldtodo\todo
\renewcommand{\todo}[1]{\Oldtodo[inline]{#1}}
\newcommand{\eric}[2][]{\Oldtodo[inline,color=blue, #1]{Eric: #2}}

% Local commands go here.

\newcommand{\note}[1]{
   \begin{tcolorbox}[colback=red!5!white, colframe=red!75!black]
      #1
   \end{tcolorbox}
}



\newcommand{\docRef}{PSTN-021}
\newcommand{\docUpstreamLocation}{\url{https://github.com/lsst-pst/pstn-021}}


\begin{document}
%% DO NOT EDIT THIS FILE. IT IS GENERATED FROM db2authors.py"
%% Regenerate using:
%%    python $LSST_TEXMF_DIR/bin/db2authors.py > authors.tex


\author[0000-0001-8018-5348]{Eric~C.~Bellm}
\affiliation{University of Washington, Dept.\ of Astronomy, Box 351580, Seattle, WA 98195, USA}

\author[0000-0001-8708-251X]{Ian~S.~Sullivan}
\affiliation{University of Washington, Dept.\ of Astronomy, Box 351580, Seattle, WA 98195, USA}


\date{\today}
\title{Alert Production with the Vera C. Rubin Observatory}
\hypersetup{pdftitle={\@title}, pdfauthor={\@author}, pdfkeywords={\@keywords}}


\begin{abstract}

Describes the LSST Alert Production pipelines, excecution environment, data products, and early performance. 

\end{abstract}



\listoftodos

\section{Introduction}

Repeated observations of the night sky enable discovery of transient, variable, and moving objects, and time-domain measurements are one of the foundational techniques of astronomical science.
The cadence of the observations, the intrinsic timescales of the phenomena, and the spatial volume probed by the observations set the rate of discovery  \eric{cite me; see also Tonry, Ofek}.
In recent years, scientists have conducted dedicated optical time-domain surveys with large CCD mosaic cameras (e.g., PanSTARRS, DECam, ZTF, HSC, Gaia, Kepler, TESS) as well as distributed telescope networks (e.g., ASAS-SN, ATLAS, \eric{more}).
These have yielded large new samples of supernovae, variable stars, active galactic nuclei, and solar system objects.
Additionally, they have uncovered the first exemplars of rare new classes of objects \eric{add some examples: FBOTS, ISOs, asteroids interior to venus, ...}.

Frequently, time-critical followup observations are necessary to classify and characterize objects discovered in time-domain surveys.
Historically, human-composed circulars or telegrams were used to dissemenate discoveries to the wider community.
The increasing rate of transient candidates as well as the desire for rapid automated followup motivated machine-readable alternatives \eric{VOEvent, GCN, SciMMA/HopSkotch, TNS}.
These trends culminated in the public ``alert stream'' paradigm employed by ZTF \eric{cite Patterson}, in which hundreds of thousands of unfiltered difference image sources are shipped along with historical lightcurves and image cutouts to third-party alert brokers \eric{cites} for classification, filtering, and followup.
This approach has enabled fully automated identification and reporting of supernova candidates.

The Legacy Survey of Space and Time (LSST) to be conducted by the Vera C.\ Rubin Observatory promises an order of magnitude \eric{confirm or cite} increase in transient discovery.
Rubin's large collecting area, wide field of view, and fast readout and slew will deliver nearly a thousand \eric{confirm} exposures across a wide swath of the Southern Hemisphere sky to unprecedented depths.
This capability motivated the development of a rapid data processing pipeline to identify and publicize time-variable phenomena in LSST images: the Rubin Alert Production System (AP).
Along with the annual Data Release Processing (DRP), these productions make use of the Rubin Science Pipelines software as well as the larger systems and infrastructure of Rubin Data Management (DM).
\eric{cites throughout}

In this paper, we describe the design, implementation, and initial performance of the Rubin Alert Production system.
\eric{outline}

%\note{See discussion on common material and timing considerations in \cite{PSTN-020}.}
%\note{This outline proceeds assuming this paper is being produced at the end of commissioning/early operations, and some form of routine alert production is happening in operations based on commissioning-derived or incremental templates.
%If only a small amount of LSSTCam commissioning data is available and is being released, only a minimal paper decribing the pipelines used to process that data would useful, and a more complete AP reference should come later.}
%\note{Information about the processing environment is presumed to be covered in \cite{PSTN-018}.
%}
%\note{SS Processing falls under Prompt Processing but detailed discussion is deferred to \cite{PSTN-025}.}



\section{Pipeline}

%\note{following LDM-151}

\subsection{Single Frame Processing}

\note{Construction of calibration products assumed described in \cite{PSTN-026}.}

ISR, PSF and Background fitting, photometric and astrometric calibration.  Likely major commonalities with DRP.

\subsection{Template Generation}

\note{Pre-DR1 this may require description independent of DRP.}

High-level discussion of DCR correction could go here.

\subsection{Image Differencing}

\subsection{Source Detection and Measurement}

Including discussion of point source, dipole, and streak detection on difference images.  

\note{Algorithmic basics could be discussed elsewhere}

\subsection{Spuriousness scoring}

Algorithm, training, and performance of the ML spuriousness score

\subsection{Source Association}

Algorithms and interface considerations with the PPDB

\note{does \cite{PSTN-045} describe the APDB/PPDB implementation?}

\subsection{Alert Generation}

Formats \& contents

\subsection{Alert Distribution}

Mechanisms, connections to community brokers

\subsection{Alert Filtering Service}

\subsection{Forced Photometry}

\subsection{Source Injection}

\subsection{Metrics}
\section{Prompt Data Products} \label{sec:data}


Summary of relevant aspects of the DPDD \citep{LSE-163}, including latency considerations and user access.

\begin{itemize}
  \item Images
  \item Prompt Catalogs and the PPDB
  \item Alerts
\end{itemize}



\section{Processing Environment}


\subsection{Prompt Processing Framework}

\subsection{Databases}

\subsection{Alert Archive}

\subsection{Metrics and Dashboards}

\subsection{Catchup Processing}
\section{Initial Performance} \label{sec:performance}



\subsection{Alert Purity} \label{sec:purity}

Raw and after ML scoring


\subsection{Alert Completeness}

\subsection{Alert Latency}

\subsection{Photometric Precision}

\subsection{Astrometric Precision}

\subsection{Association Accuracy}
\input{conclusion}

%\appendix
% Remove this when you start your paper
%\input{appendix}
% Include all the relevant bib files.
% https://lsst-texmf.lsst.io/lsstdoc.html#bibliographies
%section{References} \label{sec:bib}
\todo{figure out how to make LSST technotes have appropriate information}
\bibliographystyle{aasjournal}
\bibliography{local,lsst,lsst-dm,refs_ads,refs,books}

% Make sure lsst-texmf/bin/generateAcronyms.py is in your path
%\section{Acronyms} \label{sec:acronyms}
%\addtocounter{table}{-1}
\begin{longtable}{p{0.145\textwidth}p{0.8\textwidth}}\hline
\textbf{Acronym} & \textbf{Description}  \\\hline

AP & Alert Production \\\hline
APDB & Alert Production DataBase \\\hline
DCR & Differential Chromatic Refraction \\\hline
DPDD & Data Product Definition Document \\\hline
DRP & Data Release Production \\\hline
ISR & Instrument Signal Removal \\\hline
LSE & LSST Systems Engineering (Document Handle) \\\hline
LSST & Large Synoptic Survey Telescope \\\hline
PPDB & Prompt Products DataBase \\\hline
PSF & Point Spread Function \\\hline
PSTN & Project Science Technical Note \\\hline
\end{longtable}


\end{document}
